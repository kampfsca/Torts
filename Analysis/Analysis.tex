\documentclass{article}

\usepackage{arxiv}

\usepackage[utf8]{inputenc} % allow utf-8 input
\usepackage[T1]{fontenc}    % use 8-bit T1 fonts
\usepackage{lmodern}        % https://github.com/rstudio/rticles/issues/343
\usepackage{hyperref}       % hyperlinks
\usepackage{url}            % simple URL typesetting
\usepackage{booktabs}       % professional-quality tables
\usepackage{amsfonts}       % blackboard math symbols
\usepackage{nicefrac}       % compact symbols for 1/2, etc.
\usepackage{microtype}      % microtypography
\usepackage{graphicx}

\title{Preliminary Analysis of Galapagos Tortoise Movement Patterns}

\author{
    Andy Kampfschulte
   \\
    Spatial Sciences Institute \\
    University of Southern California \\
  Los Angeles, CA \\
  \texttt{\href{mailto:kampfsch@usc.edu}{\nolinkurl{kampfsch@usc.edu}}} \\
  }


% tightlist command for lists without linebreak
\providecommand{\tightlist}{%
  \setlength{\itemsep}{0pt}\setlength{\parskip}{0pt}}


% Pandoc citation processing
\newlength{\cslhangindent}
\setlength{\cslhangindent}{1.5em}
\newlength{\csllabelwidth}
\setlength{\csllabelwidth}{3em}
\newlength{\cslentryspacingunit} % times entry-spacing
\setlength{\cslentryspacingunit}{\parskip}
% for Pandoc 2.8 to 2.10.1
\newenvironment{cslreferences}%
  {}%
  {\par}
% For Pandoc 2.11+
\newenvironment{CSLReferences}[2] % #1 hanging-ident, #2 entry spacing
 {% don't indent paragraphs
  \setlength{\parindent}{0pt}
  % turn on hanging indent if param 1 is 1
  \ifodd #1
  \let\oldpar\par
  \def\par{\hangindent=\cslhangindent\oldpar}
  \fi
  % set entry spacing
  \setlength{\parskip}{#2\cslentryspacingunit}
 }%
 {}
\usepackage{calc}
\newcommand{\CSLBlock}[1]{#1\hfill\break}
\newcommand{\CSLLeftMargin}[1]{\parbox[t]{\csllabelwidth}{#1}}
\newcommand{\CSLRightInline}[1]{\parbox[t]{\linewidth - \csllabelwidth}{#1}\break}
\newcommand{\CSLIndent}[1]{\hspace{\cslhangindent}#1}

\begin{document}
\maketitle


\begin{abstract}

\end{abstract}

\keywords{
    Latent Class
   \and
    Tortoises
   \and
    Animal Movement
  }

\hypertarget{introduction}{%
\section{Introduction}\label{introduction}}

Data are presented for the tracking of movement of Galpagos Tortoises.
There are tortoises spread across several islands, leading research to
wonder how behavior of these animals differs from place to place. Here,
going off of an influential movement ecology paper, I examine the
movement strategies of 24 tortoises across to Islands to to (1) see how
movement differs between these islands, (2) evaluate underlying latent
states over time, by individual, and (3) to see if previous research can
be replicated regarding the movement strategies of these tortoises.

\hypertarget{methods}{%
\section{Methods}\label{methods}}

A lot of this work is a replication of (Bastille-Rousseau et al. 2016)
\cite{bastille2016}, in which a latent class modelling framework was
used in tandem with Net Squared Displacement (NSD). First, let's discuss
NSD. NSD appears to be a common apprach used to model animal movement
patterns \cite{netdis1, netdis2, bastille2016}. Simply put, it is the
squared Euclidean distance of a given point from a starting point.
Therefore, migratory animals would be expected to have a large NSD over
time, while sedentary animals would have a relatively lower NSD. This
statistic is useful in a time-series format, examining NSD over time to
evaluate movement strategies. Migratory animals, will generally have a
double-sigmoid pattern to the NSD over time, while sedentary animals
will have no clear trend of NSD. (Bastille-Rousseau et al. 2016) uses
this NSD framework to throw into a latent state model to evaluate
movement strategies for animals at any given time. The idea is that the
movement patterns of an individual at time \(t\) can be modelled as a
mixture of 3 different latent states, and examining the transitions of
these 3 states over time can give us insight into an animals overall
movement behavior. This analytical approach was chosen because (1) it
was applied to Galapagos tortoises in (Bastille-Rousseau et al. 2016)
and was shown to be a superior method to modelling animal movement, (2)
is spatio-temporal in perspective, and (3) provides the opportunity to
incorporate meterolgical data later on.

For the remainder of this paper, states 1 \& 2 are associated with
encampment-oriented behavior, while state 3 is defined as an exploratory
movement state.

\begin{table}
\centering
\begin{tabular}[t]{llrrr}
\toprule
ID & Site & Time in 1 (\%) & Time in 2 (\%) & Time in 3 (\%)\\
\midrule
Anne & Isla Isabela (Alcedo) & 63.68 & 33.67 & 2.65\\
Bertha & Isla Espanola & 92.60 & 1.20 & 6.20\\
Bill & Isla Espanola & 70.04 & 26.91 & 3.05\\
Birgit & Isla Isabela (Alcedo) & 47.97 & 36.62 & 15.41\\
Charles & Isla Espanola & 73.19 & 22.01 & 4.80\\
Christian & Isla Isabela (Alcedo) & 72.44 & 17.26 & 10.31\\
Emma Espanola & Isla Espanola & 97.30 & 2.45 & 0.25\\
Franz' girlie & Isla Isabela (Alcedo) & 63.63 & 22.76 & 13.61\\
Greg & Isla Isabela (Alcedo) & 50.63 & 49.37 & NA\\
Hamish & Isla Espanola & 82.28 & 0.64 & 17.08\\
Isabela & Isla Isabela (Alcedo) & 64.58 & 13.51 & 21.91\\
James & Isla Espanola & 91.30 & 0.60 & 8.10\\
Jeneene & Isla Isabela (Alcedo) & 88.79 & 5.70 & 5.50\\
Kiki & Isla Espanola & 50.05 & 49.51 & 0.44\\
Martin & Isla Isabela (Alcedo) & 67.28 & 21.31 & 11.41\\
Miriam & Isla Espanola & 96.00 & 4.00 & NA\\
Nathalie & Isla Espanola & 81.64 & 9.55 & 8.80\\
Randal & Isla Espanola & 91.55 & 8.45 & NA\\
Skandar & Isla Espanola & 96.80 & 1.35 & 1.85\\
Skandar's Girlfriend & Isla Espanola & 82.09 & 15.91 & 2.00\\
Sparkey & Isla Isabela (Alcedo) & 90.60 & 4.65 & 4.75\\
Spikey & Isla Isabela (Alcedo) & 78.39 & 14.01 & 7.60\\
Walter & Isla Espanola & 87.29 & 6.80 & 5.90\\
Zelfa & Isla Espanola & 92.85 & 7.00 & 0.15\\
\bottomrule
\end{tabular}
\end{table}

I thought it would be a good idea to compare the movement
classifications of as many tortoises as I could, and see if movement
patterns differed across Islands. From the available movement data,
Espanola Island - the island of interest - had 22 available tortoises
with movement data - one of which consisted of only a few observations
and was discarded, while the ecologically similar Isabela Island had 11.
This provided a total of 30 individuals to perform latent class models
on NSD values.

\begin{table}

\caption{\label{tab:unnamed-chunk-6}Deviance and Convergence Statistics for Latent Class. The main take away from this table is the Gelman Diagnositic ($\hat{R}$) is below the standard convergence criteria of 1.1}
\centering
\begin{tabular}[t]{llrrrr}
\toprule
ID & Site & Mean & SD & Median & $\hat{R}$\\
\midrule
Anne & Isla Isabela (Alcedo) & -7245.358 & 5.188575 & -7245.821 & 1.001459\\
Bertha & Isla Espanola & -8777.448 & 10.941307 & -8779.812 & 1.000639\\
Bill & Isla Espanola & -12362.024 & 21.276876 & -12363.005 & 1.002606\\
Birgit & Isla Isabela (Alcedo) & -7664.203 & 4.504521 & -7664.990 & 1.000714\\
Charles & Isla Espanola & -5748.869 & 11.727634 & -5749.958 & 1.001662\\
Christian & Isla Isabela (Alcedo) & -17779.386 & 4.021877 & -17780.210 & 1.000839\\
Emma Espanola & Isla Espanola & -10998.851 & 15.833726 & -11000.149 & 1.000847\\
Franz' girlie & Isla Isabela (Alcedo) & -4033.782 & 5.675079 & -4034.530 & 1.002420\\
Greg & Isla Isabela (Alcedo) & -2410.271 & 4.709368 & -2410.864 & 1.003427\\
Hamish & Isla Espanola & -6913.631 & 10.277682 & -6915.049 & 1.001202\\
Isabela & Isla Isabela (Alcedo) & -13231.166 & 3.871636 & -13231.967 & 1.000732\\
James & Isla Espanola & -18199.507 & 15.332074 & -18199.858 & 1.001161\\
Jeneene & Isla Isabela (Alcedo) & -21040.020 & 3.371190 & -21040.715 & 1.000688\\
Kiki & Isla Espanola & -4882.839 & 16.696270 & -4883.568 & 1.001597\\
Martin & Isla Isabela (Alcedo) & -6259.532 & 3.521871 & -6260.291 & 1.001208\\
Miriam & Isla Espanola & -3634.700 & 4.807952 & -3635.621 & 1.001147\\
Nathalie & Isla Espanola & -14718.098 & 10.318705 & -14718.148 & 1.002844\\
Randal & Isla Espanola & -2716.524 & 4.032184 & -2717.209 & 1.000585\\
Skandar & Isla Espanola & -10109.461 & 13.271079 & -10110.728 & 1.001012\\
Skandar's Girlfriend & Isla Espanola & -4064.866 & 17.854042 & -4066.703 & 1.001207\\
Sparkey & Isla Isabela (Alcedo) & -20650.052 & 4.224539 & -20650.688 & 1.002913\\
Spikey & Isla Isabela (Alcedo) & -4954.766 & 3.827587 & -4955.484 & 1.002036\\
Walter & Isla Espanola & -4868.579 & 6.052226 & -4869.381 & 1.000730\\
Zelfa & Isla Espanola & -5487.994 & 12.195589 & -5486.981 & 1.002568\\
\bottomrule
\end{tabular}
\end{table}

The \({\sf{R}}\) package \texttt{lsmnsd} was used to explore the latent
states for the 30 individuals. This uses Markov Chain Monte Carlo
methods to optimise the fit for each latent state. A total 10,000O
sample iterations were run to get the transition matrix to converge,
50\% of the iterations were discarded as burn-in, and a thinning
interval of 1/5 was applied for a total 4,000 viable simulations per
individual. This took approximately 90 minutes to run in parallel using
an 11th generation Intel i7 processor with 16 cores. Of the 30, 6
individual tortoise models were unable to converge properly, and were
discarded. Leaving a final analysis of 24 tortoises, 14 from Espanlola
Island, and 10 from Isabela Island.

\hypertarget{results}{%
\section{Results}\label{results}}

Analysis was hampered due to computational bottlenecks, so time steps
were limited to 2000 for each individual tortoise.

\hypertarget{difference-in-states-between-islands}{%
\subsection{Difference in States Between
Islands}\label{difference-in-states-between-islands}}

\begin{figure}
\centering
\includegraphics[scale=.6]{"../plots/class.png"}
\caption{Results of the latent state model/net squared displacement analysis to classify movement patterns.}
\label{fig:class}
\end{figure}

Figure \ref{fig:class} shows the density distribution of the three
latent states for each island. While the shapes of the distribution are
similar, there appears to be a much larger density of the first state,
associated with encampment behavior on Espanola, while Isabela Island
has a higher density of values at state 3, indicating more exploratory
movement patterns.

\hypertarget{differences-in-transition-patterns-between-islands}{%
\subsection{Differences in Transition Patterns Between
Islands}\label{differences-in-transition-patterns-between-islands}}

\begin{figure}
\centering
\includegraphics[scale=.6]{"../plots/shifts.png"}
\caption{Boxplots of the frequency of Movement state shifts by Island.}
\label{fig:shifts}
\end{figure}

Exploring the frequency of latent shifts is where things get interesting
and the inter-island (and inter-individual) differences become clearer
(figure \ref{fig:shifts}. For Isabela Island, many of the latent shift
occur going from states 2 to 3 and states 3 to 1. The fact that state 3
is present in the most frequently observed shifts is evidence that these
tortoises are more migratory in their movement strategy.

Espanola Island, on the other hand, has the highest frequency of laten
shifts going between states 1 and 2 (in both directions). This is
evidence of a more prevalent encampment movement strategy, where
individuals seldom migrate or deviate far from an initial start point.

\begin{figure}
\centering
\includegraphics[scale=.6]{"../plots/NSD2.png"}
\caption{Net Squared Displacement (NSD) of individual tortoises over time. The lines are coloured by the assigned movement class of the individual within each time step.}
\label{fig:eh}
\end{figure}

On an individual level figure \ref{fig:eh} shows NSD values over time,
coloured by latent state assignment for each individual. We can see that
many of the Espanola tortoises have low NSD values to begin with, while
Isabela tortoises have unique migratory and nomadic patterns.

\hypertarget{discussion}{%
\section{Discussion}\label{discussion}}

While taking some time to understand on an intuitive level, I found the
latent state approach to assessing movement strategies a reasonable
approach. This work, depsite being incomplete as of this writing,
already serves as an extension of (Bastille-Rousseau et al. 2016) by
increasing the sample size of tortoises from 8 to 24, and also
reinforces its findings that tortoises on Espanola Island do not exhibit
migratory behavior like some individuals do on Isabela Island. A close
examination of figure \ref{fig:eh} shows that NSD is often most erratic
in Espanola tortoises, while many of the NSD trends that are the most
clear-cut are on Isabela.

\hypertarget{ideal-next-steps}{%
\section{Ideal Next Steps}\label{ideal-next-steps}}

This was an immense amount of legwork just to derive a response variable
of movement type. Continuing off of this work, it would be prudent to
take time to rexamine the data inputs and analysis. Much of this work
was truncated to expedite results. For example, only the first 2,000
time stamps were taken for each individual to reduce computational
constraints. Only Isabella Island was selected s a comparison for
Espanola Island for this reason as well. Revisiting these decisions
could lead to a more rich, verdant, analysis. Second, this work should
be continued; this only the first phase. Incorporating meterological
data to compare against these observed movement patterns on different
islands would be a worthy endpoint for this analysis. There is much
opportunity to incorporate the weather readings from island weather
stations and on the individual GPS devices to place these individuals
within a meterological context, and explore what role weather plays on
movement patterns.

\hypertarget{a-note-from-the-author}{%
\section{A Note From the Author}\label{a-note-from-the-author}}

To close this out, I'd like to break a steadfast rule in presenting
one's own research: \emph{never undersell}. This semester was without
doubt a bit of a roller coaster, and chaotic at times, so before I start
breaking rules, let me offer my sincere thanks for the support and
backing of SSI through the ups and downs of the semester.

Many thanks aside, I'd like to be frank and admit that I don't think
this is my best work. I grossly underestimated the complexity of the
data, and even moreso, figuring out how to frame a research question
from the data. While this work was in some ways and analytical challenge
(latent states, transition matrices, etc), the biggest challenge was
pivoting from my little world of public health statistics and data, to
learning the topography of movement ecology research. The data are
somewhat similiar, but the questions and mindset seem to be on opposite
poles. That, and a rather severe case of COVID in the first week of
December, have left me wishing I could've gotten further.

\hypertarget{refs}{}
\begin{CSLReferences}{1}{0}
\leavevmode\vadjust pre{\hypertarget{ref-bastille2016}{}}%
Bastille-Rousseau, Guillaume, Jonathan R Potts, Charles B Yackulic,
Jacqueline L Frair, E Hance Ellington, and Stephen Blake. 2016.
{``Flexible Characterization of Animal Movement Pattern Using Net
Squared Displacement and a Latent State Model.''} \emph{Movement
Ecology} 4 (1): 1--12.

\end{CSLReferences}

\bibliographystyle{unsrt}
\bibliography{references.bib}


\end{document}
